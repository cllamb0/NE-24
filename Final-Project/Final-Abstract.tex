\documentclass[12pt]{article}
\usepackage[top=1in, bottom=1in, left=1in, right=1in]{geometry}
\usepackage{hyperref}
\author{Chris Lamb}
\title{NE 24 - Final Project Abstract}
\date{28 April, 2017}

\begin{document}
%Assignment Header
\begin{center}
\textbf{NE 24 - Final Project Abstract \\ Chris Lamb \\ 28 April, 2017}
\end{center}
\noindent\rule{17cm}{1pt}

%Abstract Title
\begin{center}
\textbf{Plinth and FreedomBox / Contributions to a GitHub Source Code}
\rule{14cm}{0.4pt}
\end{center}

%Bulk of the Abstract
Following trend of Moore's Law, computers have been growing at a constant rate since their inception. Thus the amount of storage needed to house all this information has also increased at a near identical rate. Along with these increases in storage need comes the need for increased security of that stored information and FreedomBox is an open source way to store your information. FreedomBox is a way for consumers to reliably store whatever information they need, it takes away the outside companies and allows the user to take full control of their personal data. Plinth is a software that goes with FreedomBox to help the user with setting up their personal cloud storage. Plinth allows the user operation to possibly be as simple as the operation of a smartphone. For my final project, I helped work on the source code itself and solved one of the small issues that they were having with the overall documentation of Plinth. I took care of an issue in which a technical part of the installation instructions was not properly worded. Had users been following the directions directly, then they would not have Plinth installed correctly on their computer. After changing that little bit of documentation to the correct version, I subbmitted my first pull request to the owners of the source code and got it accepted. This project was mainly an exercise on working with source code and getting comfortable working with GitHub. In that aspect I think that it was a success helping out with a source code that I find would possibly be useful to me in the future. If new coders are interested in doing something similar to what I did then I would definitely suggest for them to look through different repositories to find something they are interested in. Once they find something, look through the issues tab of the repository and see if there is any issue that they think they can solve. From there on, it's up to the coder to contribute whatever they like!

\begin{center}
Also, here's a link to the pull request that shows my final project: \href{https://github.com/freedombox/Plinth/pull/765}{https://github.com/freedombox/Plinth/pull/765}
\end{center}
\end{document}