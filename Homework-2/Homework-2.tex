\documentclass[12pt]{article}
\usepackage[top=0.75in, bottom=0.75in, left=1in, right=1in]{geometry}
\usepackage{hyperref}
\author{Chris Lamb}
\date{13 February, 2017}
\title{NE-24 Homework \#2}

\begin{document}
\begin{center}
%Assignment Header
\textbf{NE 24 - Homework \#2 \\ Chris Lamb \\ 13 February, 2017}
\end{center}
\noindent\rule{17cm}{1pt}

%Section 1 Title
\vskip.4in
\begin{center}
\underline{\textbf{Section 1: Creating a Repository}}
\end{center}
\vskip.25in

\begin{enumerate}
%Question 1, Section 1
\item Dracula starts a new project, \emph{moons}, related to his \emph{planets} project. Despite Wolfwoman’s concerns, he enters the following sequence of commands to create one Git repository inside another:
\begin{flushleft}
\texttt{cd             \# return to home directory \\ mkdir planets  \# make a new directory planets \\ cd planets     \# go into planets \\ git init       \# make the planets directory a Git repository \\ mkdir moons    \# make a sub-directory planets/moons \\ cd moons       \# go into planets/moons \\ git init       \# make the moons sub-directory a Git repository}
\end{flushleft}
Why is it a bad idea to do this? How can Dracula "undo" his last \emph{git init}?
\begin{center}
\textsf{\underline{Answer:} It’s a bad idea to do this because when the repositories are nested inside of each other, they mess with each other and cause errors. To fix this error, you can remove the .git repository with rm –rf moons/.git}
\end{center}
\end{enumerate}

%Section 2 Title
\vskip.4in
\begin{center}
\underline{\textbf{Section 2: Tracking Changes}}
\end{center}
\vskip.25in 

\begin{enumerate}
%Question 1, Section 2
\item Which command(s) below would save the changes of \emph{myfile.txt} to my local Git repository?
\begin{flushleft}
\texttt{\$ git commit -m "my recent changes"}
\end{flushleft}
\begin{flushleft}
\texttt{\$ git init myfile.txt \\ \$ git commit -m "my recent changes"}
\end{flushleft}
\begin{flushleft}
\texttt{\$ git add myfile.txt \\ \$ git commit -m "my recent changes"}
\end{flushleft}
\begin{flushleft}
\texttt{\$ git commit -m myfile.txt "my recent changes"}
\end{flushleft}
\begin{center}
\textsf{\underline{Answer:} \$ git add myfile.txt \\ \$ git commit -m "my recent changes"}
\end{center}

%Question 2, Section 2
\item Create a new Git repository on your computer called \emph{bio}. Write a three-line biography for yourself in a file called \emph{me.txt}, commit your changes, then modify one line, add a fourth line, and display the differences between its updated state and its original state.
\begin{center}
\textsf{\underline{Answer:} Here's a screenshot of the answer: \href{http://i.imgur.com/xvGHjaE.png}{http://i.imgur.com/xvGHjaE.png}}
\end{center}
\end{enumerate}

%Section 3 Title
\vskip.4in
\begin{center}
\underline{\textbf{Section 3: Exploring History}}
\end{center}
\vskip.25in 

\begin{enumerate}
%Question 1, Setion 3
\item Jennifer has made changes to the Python script that she has been working on for weeks, and the modifications she made this morning “broke” the script and it no longer runs. She has spent $\sim$1hr trying to fix it, with no luck\ldots
Luckily, she has been keeping track of her project’s versions using Git! Which commands below will let her recover the last committed version of her Python script called \emph{data\_cruncher.py}?
\begin{flushleft}
\texttt{\textbf{1.} \$ git checkout HEAD}
\end{flushleft}
\begin{flushleft}
\texttt{\textbf{2.} \$ git checkout HEAD data\_cruncher.py}
\end{flushleft}
\begin{flushleft}
\texttt{\textbf{3.} \$ git checkout HEAD$\sim$1 data\_cruncher.py}
\end{flushleft}
\begin{flushleft}
\texttt{\textbf{4.} \$ git checkout <unique ID of last commit> data\_cruncher.py}
\end{flushleft}
\begin{flushleft}
\texttt{\textbf{5.} Both 2 \& 4}
\end{flushleft}
\begin{center}
\textsf{\underline{Answer:} Both 2 \& 4}
\end{center}
\end{enumerate}

%Section 4 Title
\vskip.4in
\begin{center}
\underline{\textbf{Section 4: Ignoring Things}}
\end{center}
\vskip.25in 

\begin{enumerate}
%Question 1, Section 4
\item So this section wanted me to create a .gitignore file within a repository on GitHub but if you look within this entire NE-24 repository, you'll see that I created one for the LaTeX files in the Homework folders so I think that's good enough.
\begin{center}
\textsf{\underline{Answer:} See the core section of my NE-24 repository.}
\end{center}
\end{enumerate}

%Section 5 Title 
\vskip.4in
\begin{center}
\underline{\textbf{Section 5: Remotes in GitHub}}
\end{center}
\vskip.25in 

\begin{enumerate}
%Question 1, Section 5
\item Create a repository on GitHub, clone it, add a file, push those changes to GitHub, and then look at the timestamp of the change on GitHub. How does GitHub record times, and why?
\begin{center}
\textsf{\underline{Answer:} GitHub records times by the amount of time ago the commit was uploaded. It shows it that why so that we can view the most recently changed files.}
\end{center}
\end{enumerate}
\end{document}