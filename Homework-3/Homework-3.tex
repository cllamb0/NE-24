\documentclass[12pt]{article}
\usepackage[top=0.75in, bottom=0.75in, left=1in, right=1in]{geometry}
\usepackage{hyperref}
\author{Chris Lamb}
\date{20 February, 2017}
\title{NE-24 Homework \#3}

\begin{document}
\begin{center}
%Assignment Header
\textbf{NE 24 - Homework \#3 \\ Chris Lamb \\ 20 February, 2017}
\end{center}
\noindent\rule{17cm}{1pt}

%Section 1 Title
\vskip.4in
\begin{center}
\underline{\textbf{Section 1: Conflicts}}
\end{center}
\vskip.25in

\begin{enumerate}
%Question 1, Section 1
\item Clone the repository created by your instructor. Add a new file to it, and modify an existing file (your instructor will tell you which one). When asked by your instructor, pull her changes from the repository to create a conflict, then resolve it.
\begin{center}
\textsf{\underline{Answer:} Here's a screenshot of before the resolution of the merge conflict: \href{http://i.imgur.com/Uur7Jfx.png}{http://i.imgur.com/Uur7Jfx.png} \\ and here's a screenshot of after the resolution of the conflict: \href{http://i.imgur.com/camDWKf.png}{http://i.imgur.com/camDWKf.png}}
\end{center}

%Question 2, Section 1
\item What does Git do when there is a conflict in an image or some other non-textual file that is stored in version control?
\begin{center}
\textsf{\underline{Answer:} If there is a conflict in for example an image, then the image cannot be edited in the same way that a text file can. Instead of editing the photos, you will have to choose which version of the photo that you want to keep. Either way, you would use \emph{git checkout} to choose which version of the photo that you want to be in the remote.}
\end{center}
\end{enumerate}

%Section 2 Title
\vskip.4in
\begin{center}
\underline{\textbf{Section 2: Open Science}}
\end{center}
\vskip.25in

\begin{enumerate}
%Question 1, Section 2
\item After reading through the section titled: \textsc{Open Science}, write down two things that you learned about doing science reproducibly.
\begin{center}
\textsf{\underline{Answer:} \begin{itemize} \item One way to make your work reproducible is to post the details of the \\ experiment/lab/research to be public so that everyone can do the same process.\\ \item Another way to make the code open source with good commenting so that everyone can look through and make sure to repeat it the exact same way that you did it. \end{itemize}}
\end{center} 
\end{enumerate}
\end{document}