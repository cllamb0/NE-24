\documentclass[12pt]{article}
\usepackage{alltt}
\author{Chris Lamb}
\date{2 February, 2017}
\title{NE-24 Homework 1}

\begin{document}
\begin{center}

%Assignment Header
\textbf{NE 24 - Homework \#1 \\ Chris Lamb \\ 2 February, 2017}
\end{center}
\noindent\rule{14cm}{1pt}

%Section 1 Title
\vskip.4in
\begin{center}
\textbf{Section 1: Shell Files and Directories} 
\end{center}
\vskip.25in

\begin{enumerate}
%Question 1, Section 1
\item If \emph{pwd} displays \emph{/Users/thing}, what will \emph{ls ../backup} display?
\begin{center}
\textsf{\underline{Answer:} ../backup: No such file or directory}
\end{center}

%Question 2, Section 1
\item For a hypothetical filesystem location of \emph{/home/amanda/data/}, select each of the below commands that Amanda could use to navigate to her home directory, which is \emph{/home/amanda}
\begin{center}
\textsf{\underline{Answer:} cd /home/amanda \textit{or} cd \~{} \textit{or}  cd ..  \textit{or}  cd}
\end{center}

%Question 3, Section 1
\item If \emph{pwd} displays \emph{/Users/backup}, and \emph{-r} tells \emph{ls} to display things in reverse order, what command will display: \\ \emph{pnas-sub/ pnas-finals/ original/}
\begin{center}
\textsf{\underline{Answer:} Either ls -r -F or ls -r -F /users/backup but not ls pwd}
\end{center}

%Question 4, Section 1
\item What does the command \emph{cd} without a directory name do?
\begin{center}
\textsf{\underline{Answer:} It changes the working directory to the user's home directory.}
\end{center}

%Question 5, Section 1
\item What does the command \emph{ls} do when used with the \emph{-s} and \emph{-h} arguments?
\begin{center}
\textsf{\underline{Answer:} When –s is used with ls, it shows the sizes of the files in blocks and when –h is used with ls, it turns the sizes in a human readable format.}
\end{center}
\end{enumerate}

%Section 2 Title
\vskip.4in
\begin{center}
\textbf{Section 2: Creating Things}
\end{center}
\vskip.25in

\begin{enumerate}
%Question 1, Section 2
\item Suppose that you created a \emph{.txt} file in your current directory to contain a list of the statistical tests you will need to do to analyze your data, and named it: \emph{statstics.txt} \\ \\ After creating and saving this file you realize you misspelled the filename! You want to correct the mistake, which of the following commands could you use to do so?
\begin{center}
\textsf{\underline{Answer:} cp statstics.txt statistics.txt}
\end{center}

%Question 2, Section 2
\item What is the output of the closing \emph{ls} command in the sequence shown below?
\begin{flushleft}
\texttt {\$ pwd \\ /Users/jamie/data \\ \$ ls \\ proteins.dat \\ \$ mkdir recombine \\ 
\$ mv proteins.dat recombine \\ \$ cp recombine/proteins.dat ../proteins-saved.dat \\ \$ ls}
\end{flushleft}
\begin{center}
\textsf{\underline{Answer:} recombine}
\end{center}

%Question 3, Section 2
\item Jamie is working on a project and she sees that her files aren’t very well organized:
\begin{flushleft}
\texttt{\$ ls -F \\ analyzed/  fructose.dat    raw/   sucrose.dat}
\end{flushleft}
The \emph{fructose.dat} and \emph{sucrose.dat} files contain output from her data analysis. What command(s) covered in this lesson does she need to run so that the commands below will produce the output shown?
\begin{flushleft}
\texttt{\$ ls -F \\ analyzed/   raw/ \\ \$ ls analyzed \\ fructose.dat    sucrose.dat}
\end{flushleft}
\begin{center}
\textsf{\underline{Answer:} mv fructose.dat /analyzed \\ mv sucrose.dat /analyzed}
\end{center}

%Question 4, Section 2
\item What does \emph{cp} do when given several filenames and a directory name, as in:
\begin{flushleft}
\texttt{\$ mkdir backup \\ \$ cp thesis/citations.txt thesis/quotations.txt backup}
\end{flushleft}
What does \emph{cp} do when given three or more filenames, as in:
\begin{flushleft}
\texttt{\$ ls -F \\ intro.txt    methods.txt    survey.txt \\ \$ cp intro.txt methods.txt survey.txt}
\end{flushleft}
\begin{center}
\textsf{\underline{Answer:} When given multiple filenames, it will copy and move those multiple files into the chosen directory. The second command gives an error because it is trying to move the files to a folder but no folder was given.}
\end{center}

%Question 5, Section 2
\item The command \emph{ls -R} lists the contents of directories recursively, i.e., lists their sub-directories, sub-sub-directories, and so on in alphabetical order at each level. The command \emph{ls -t} lists things by time of last change, with most recently changed files or directories first. In what order does \emph{ls -R -t} display things?
\begin{center}
\textsf{\underline{Answer:} Typing ls -R -t will give the full directories in order alphabetically and in order of most recently edited.}
\end{center}
\end{enumerate}

%Section 3 Title
\vskip.4in
\begin{center}
\textbf{Section 3: Pipes and Filters}
\end{center}
\vskip.25in

\end{document}